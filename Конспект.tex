\documentclass[12pt]{article}
\usepackage[utf8]{inputenc}
\usepackage[T2A]{fontenc}
\usepackage[russian, english]{babel}
\usepackage{amsthm}
\usepackage{amssymb}
\usepackage{amsmath}
\usepackage{authblk}
\usepackage{bm}
\usepackage{enumerate}
\usepackage{mathabx}
\usepackage{titling}
\usepackage{titlesec}
\usepackage{ragged2e}
\usepackage{polynom}
\usepackage{hyperref}
\usepackage[normalem]{ulem}
\usepackage[left=3cm,right=3cm, top=2cm,bottom=2cm,bindingoffset=0cm]{geometry}

\titlelabel{\thetitle.\quad}

\hypersetup{
    colorlinks=true, %set true if you want colored links
    linktoc=all,     %set to all if you want both sections and subsections linked
    linkcolor=black,  %choose some color if you want links to stand out
}
\textwidth = 16cm

\newcommand{\stkout}[1]{\ifmmode\text{\sout{\ensuremath{#1}}}\else\sout{#1}\fi}

\begin{document}

\renewcommand{\contentsname}{Содержание}
\title{Алгоритмы компьютерной алгебры}
\author{Конспект лекций}
\date{2019}
\maketitle
\newpage

\tableofcontents
\newpage

\theoremstyle{plain}
\newtheorem{thm}{Теорема}
\newtheorem{defn}{Определение}
\newtheorem{examp}{Пример}
\newtheorem{lem}{Лемма}
\newtheorem{sled}{Следствие}
\newtheorem*{sl*}{Следствие}

\section{Лекция 1.}
Предмет изучения компьютерной алгебры - точные вычисления. Рассматриваются именно алгоритмы точного, а не приближенного вычисления, как в вычислительной математике. Эти алгоритмы лежат в основе математических пакетов MATLAB, Mathematica. Основной объект исследованй - числовые системы с точными вычислениями.

\subsection{Основные факты из теории многочленов}
\begin{defn}
\textbf{Числовым полем} называется множество $\mathrm{F} \subset \mathbb{C}$, если:
\begin{enumerate}
\item $0, 1 \in F$
\item $\mathrm{|F| \ge 2}$
\item $\forall a,b \in F: a \pm b, ab \in F; b \ne 0, \dfrac{a}{b} \in F.$
\end{enumerate}
\end{defn}

\begin{examp}
Числовые поля - $\mathbb{C}, \mathbb{R}, \mathbb{Q}, \{a + b\sqrt{2},~a,b \in \mathbb{Q}\}$
\end{examp}
Множество многочленов над полем рациональных чисел обозначается как $\mathbb{Q}[x]$, над целыми $-~\mathbb{Z}[x]$,
над произвольным числовым полем $F$ $-~F[x].$

\begin{defn}
Многочлен $f(x) \in \mathrm{F}[x]$, отличный от константы, называют \textbf{ приводимым} над полем $\mathrm{F}$, если он допускает представление вида $f(x) = \varphi(x) \psi(x)$, где $\varphi(x), \psi(x) \in \mathrm{F}[x]$ и $deg~\varphi,deg~\psi < deg~f$, и \textbf{ неприводимым}, если он не допускает такого разложения (то есть один из многочленов $\varphi, \psi$ является константой).
\end{defn}

\begin{enumerate}
\item $deg f = 1$.
Пусть $f$ допускает разложение: $f(x) = \varphi(x) \psi(x)$. 
\item $\mathrm{|F| \ge 2}$
\item $\forall a,b \in F: a \pm b, ab \in F; b \ne 0, \dfrac{a}{b} \in F.$
\end{enumerate}

\end{document}