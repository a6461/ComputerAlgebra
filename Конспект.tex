\documentclass[12pt]{article}
\usepackage[utf8]{inputenc}
\usepackage[T2A]{fontenc}
\usepackage[russian, english]{babel}
\usepackage{amsthm}
\usepackage{amssymb}
\usepackage{amsmath}
\usepackage{authblk}
\usepackage{bm}
\usepackage{centernot}
\usepackage{enumerate}
\usepackage{mathabx}
\usepackage{titling}
\usepackage{titlesec}
\usepackage{ragged2e}
\usepackage{polynom}
\usepackage{hyperref}
\usepackage[normalem]{ulem}
\usepackage[left=3cm,right=3cm, top=2cm,bottom=2cm,bindingoffset=0cm]{geometry}

\titlelabel{\thetitle.\quad}

\hypersetup{
    colorlinks=true, %set true if you want colored links
    linktoc=all,     %set to all if you want both sections and subsections linked
    linkcolor=black,  %choose some color if you want links to stand out
}
\textwidth = 16cm

\newcommand{\stkout}[1]{\ifmmode\text{\sout{\ensuremath{#1}}}\else\sout{#1}\fi}
\hyphenation{ко-эф-фи-ци-ент со-мно-жи-те-лей пер-вой вы-чис-ле-ни-я-ми мно-го-чле-нов
про-из-ве-де-ни-е мно-жи-те-ля раз-ло-же-ни-е мно-го-чле-на де-ли-те-ля зна-че-ни-е
ко-эф-фи-ци-ен-та-ми про-стей-шее рас-смат-ри-ва-ют-ся вы-чис-ли-тель-ной}

\begin{document}

\renewcommand{\contentsname}{Содержание}
\title{Алгоритмы компьютерной алгебры}
\author{Конспект лекций}
\date{2019}
\maketitle
\newpage

\tableofcontents
\newpage

\theoremstyle{plain}
\newtheorem{thm}{Теорема}
\newtheorem*{rev*}{Обратное утверждение}
\newtheorem{defn}{Определение}
\newtheorem{examp}{Пример}
\newtheorem{lem}{Лемма}
\newtheorem{sled}{Следствие}
\newtheorem*{sl*}{Следствие}
\newtheorem{predp}{Предположение}
\newtheorem*{task*}{Задача}

\def\MYdeg{\mathop{\rm deg}\limits}

\section{Лекция 1.}
\hspace{0.6cm} Предмет изучения компьютерной алгебры - точные вычисления. Рассматриваются именно алгоритмы точного, а не приближенного вычисления, как в вычислительной математике. Эти алгоритмы лежат в основе математических пакетов MATLAB, Mathematica. Основной объект исследований - числовые системы с точными вычислениями.

\subsection{Основные факты из теории многочленов}
\begin{defn}
\textbf{Числовым полем} называется множество $\mathrm{F} \subset \mathbb{C}$, если:
\begin{enumerate}
\item $0, 1 \in F$,
\item $\mathrm{|F| \ge 2}$,
\item $\forall a,b \in F: a \pm b,~ab \in F;~b \ne 0, \dfrac{a}{b} \in F.$
\end{enumerate}
\end{defn}

\begin{examp}
Числовые поля - $\mathbb{C}, \mathbb{R}, \mathbb{Q}, \{a + b\sqrt{2},~a,b \in \mathbb{Q}\}$
\end{examp}
Множество многочленов над полем рациональных чисел обозначается как $\mathbb{Q}[x]$, над целыми $-~\mathbb{Z}[x]$,
над произвольным числовым полем $F$ $-~F[x].$

\begin{defn}
Многочлен $f(x) \in \mathrm{F}[x]$, отличный от константы, называют \textbf{ приводимым} над полем $\mathrm{F}$, если он допускает представление вида $f(x) = \varphi(x) \psi(x)$, где $\varphi(x), \psi(x) \in \mathrm{F}[x]$ и $\MYdeg \varphi, \MYdeg \psi < \MYdeg f$, и \textbf{ неприводимым}, если он не допускает такого разложения (то есть один из многочленов $\varphi, \psi$ является константой).
\end{defn}

\begin{enumerate}
\item $\MYdeg f = 1$.
Пусть $f$ допускает разложение: $f(x) = \varphi(x) \psi(x)$. 
$$\MYdeg_{= 0}\varphi, \MYdeg_{= 0}\psi < \MYdeg f \Rightarrow \MYdeg f = 0.$$
Полученное противоречие доказывает неприводимость любого многочлена первой степени.
\item Пусть $\MYdeg f > 1$ и $f(\alpha) = 0, \alpha \in F.$
$$(x - \alpha)~|~f(x) \Rightarrow \exists g(x): f(x) = (x - \alpha)g(x).$$
$$\MYdeg (x - \alpha) = 1 < \MYdeg f.$$
$$\MYdeg g = \MYdeg f - 1 < \MYdeg f.$$
Если многочлен $f$ имеет корень в поле $F$, то $f$ приводим над полем $F$. 
\end{enumerate}

\begin{rev*}
Если многочлен $f \in F[x]$ степени 2 или 3 приводим над полем $F$, то он имеет в этом поле корень. 
\end{rev*}
\begin{proof}[\textbf{Доказательство}]
Допустим, многочлен приводим, следовательно, $f(x) = \varphi(x) \psi(x)$.
$$\MYdeg \varphi, \MYdeg \psi < \MYdeg f \Rightarrow \MYdeg \varphi = 1 \text{ или} \MYdeg \psi = 1.$$
Допустим, $\varphi(x) = ax + b, a \neq 0 \Rightarrow \alpha = -\dfrac{b}{a}, \alpha \in F.$
\end{proof}

\begin{examp}
\mbox{}
\begin{enumerate}
\item $f(x) = x^2 - 1 = (x - 1)(x + 1).$
Многочлен приводим над полями $\mathbb{Q}, \mathbb{R}, \mathbb{C}$.
\item $f(x) = x^2 - 2 \in \mathbb{Q}[x]$. У него нет рациональных корней, следовательно, он неприводим над $\mathbb{Q}$. Но $f(x) = (x - \sqrt{2})(x + \sqrt{2}) \Rightarrow f(x)$ приводим над $\mathbb{R}$.
\item $f(x) = x^2 + 1$ неприводим над $\mathbb{Q}$ и $\mathbb{R}$. Но $f(x) = (x - i)(x + i) \Rightarrow f(x)$ приводим над $\mathbb{C}$.
\end{enumerate}
\end{examp}

Многочлены второй и третьей степени приводимы над полем $F$ тогда и только тогда, когда имеют в этом корень. Для многочленов степени, больше чем 3, данное утверждение не является справедливым.

\begin{examp}
$f(x) = (x^2 + 1)^2 \in \mathbb{R}[x]$ не имеет действительных корней, но приводим.
\end{examp}

\begin{defn}
Многочлен называется \textbf{нормированным}, если его старший коэффициент равен единице.
\end{defn}

\begin{thm}[Фундаментальная теорема о многочленах]
Пусть $f \in \mathrm{F}[x],$\mbox{$~deg~f \geq 1$.} Тогда $f$ допускает разложение $f(x) = a_0 \varphi_1(x) \varphi_2(x) ... \varphi_k(x)$,
 где $a_0 \in \mathrm{F},~\varphi_i \in \mathrm{F}[x]$ и любой многочлен $\varphi_i$ - нормированный и неприводимый. При этом данное разложение является единственным с точностью до порядка следования сомножителей.
\end{thm}

\subsection{Многочлены с рациональными коэффициентами}
\begin{task*}
Дан многочлен с рациональными коэффициентами. Необходимо найти разложение этого многочлена в произведение многочленов с рациональными коэффициентами.
\end{task*}

Пусть $f \in \mathbb{Q}[x].$ Если мы умножим этот многочлен на подходящее число $N$ (наименьшее общее кратное коэффициентов членов многочлена), то $Nf(x) \in \mathbb{Z}[x].$ Таким образом, приводимость $f$ равносильна приводимости $Nf$, следовательно, разложение многочлена с рациональными коэффициентами можно свести к разложению многочлена с целыми коэффициентами. 

\begin{thm}
Если многочлен $f \in \mathbb{Z}[x]$ допускает разложение в произведение многочленов с рациональными коэффициентами, то он допускает разложение в произведение многочленов тех же степеней с целыми коэффициентами.
\end{thm}

\subsubsection{Алгоритм Кронекера}
\begin{task*}
Дан многочлен $f \in \mathbb{Z}[x],~deg~f>1.$ Можно ли подобрать $u(x), v(x), ~u, v \in \mathbb{Z}[x]$ и $deg~u,~deg~v < deg~f?$
\end{task*}

\begin{predp}
Все возникающие натуральные числа можно факторизовать.
\end{predp}

\begin{predp}
Многочлен формальной степени $n$ можно найти с помощью интерполяционного многочлена по $n + 1$ точке \mbox{$x_0, x_1, ..., x_{n}$} и значениям многочлена в этих точках \mbox{$f(x_0), f(x_1), ..., f(x_n)$.}
\end{predp}

$$\begin{cases}
f(x_0) = u(x_0)v(x_0), \\
f(x_1) = u(x_1)v(x_1), \\
... \\
f(x_n) = u(x_n)v(x_n). \\
\end{cases}$$

Рассмотрим точки $x_0, x_1, ..., x_{n} \in \mathbb{Z}$.
$$\forall i \in [0, n]: f(x_i) \in \mathbb{Z} \Rightarrow u(x_i), v(x_i) \in \mathbb{Z},~deg~u = m.$$
\indentПусть все рассматриваемые точки - не корни многочлена $f$. Тогда $u(x_i)~|~f(x_i),$ $u(x_i)$ может принимать только конечное множество значений, состоящее из делителей $f(x_i)$. Коэффициенты многочлена $u$ восстанавливаются по его значениям. Далее следует непосредственная проверка того, является ли $u$ делителем $f$. Алгоритм Кронекера используется для сведения от выбора из бесконечного числа вариантов к выбору из конечного числа вариантов.  

\begin{thm}[Признак Эйзенштейна]
Пусть многочлен $$f(x) = a_0x^n + a_1x^{n-1} + ... + a_{n-1}x + a_n \in \mathbb{Z}[x],~n > 1,~a_0 \neq 0.$$ Если существует простое число $p$ такое, что $p \notdivides a_0,~p \divides a_1,~p \divides a_2,~...,~p \divides a_{n - 1}$ и $p^2 \notdivides a_n$, то $f$ неприводим над $\mathbb{Q}$.
\end{thm}

\begin{examp}
Многочлен $f(x) = x^n - 2$ не приводим над $\mathbb{Q}$ для $\forall n \geq 1$. Таким образом, существуют неприводимые многочлены над $\mathbb{Q}$ любой степени.
\end{examp}

\subsubsection{Алгоритм Евклида}

\hspace{0.6cm}Если многочлены $f, g \in F[x],~g \neq 0,$ то имеет место следующее представление: \mbox{$f(x) = g(x)h(x) + r(x)$}$,~h, r \in F[x]$ и $r = 0$ или $r \neq 0,~deg~r < deg~g.$
Если считать, что степень нулевого многочлена $r = 0$ равна $-\infty$, то можно рассматривать только вариант $deg~r < deg~g$.

\begin{defn}
Если многочлены $f, g \in F[x]$, то многочлен $\varphi \in F[x]$ называют \textbf{наибольшим общим делителем (НОД)} $f$ и $g$, если:
\begin{enumerate}
\item $\varphi(x)~|~f(x),~ \varphi(x)~|~g(x)$,
\item $\forall \psi \in F[x]: \psi(x)~|~f(x),~ \psi(x)~|~g(x) \Rightarrow \psi(x)~|~\varphi(x)$.
\end{enumerate}
\end{defn}
Можно доказать, что НОД всегда существует и находится с точностью до множителя. 

\begin{defn}
Если НОД многочленов $f(x)$ и $g(x)$  - нормированный многочлен, то он обозначается как $(f(x), g(x))$.
\end{defn}

~\

\noindent\textbf{Алгоритм Евклида. Шаг 1.} $$f(x) = g(x)h_1(x) + r_1(x).$$
$$(f(x),g(x)) = (g(x), r_1(x)),~deg~r_1 < deg~g.$$
\\
\textbf{Алгоритм Евклида. Шаг 2.} $$g(x) = r_1(x)h_2(x) + r_2(x).$$
$$(g(x),r_1(x)) = (r_1(x), r_2(x)),~deg~r_2 < deg~r_1.$$

Если степень многочлена $f$ (делимого) меньше, чем степень многочлена $g$ (делителя), то алгоритм сам поменяет их местами:
$$f(x) = g(x) \cdot 0 + f(x)$$
$$g(x) = f(x)h_1(x) + r_1(x)$$

Поскольку остаток - неотрицательный, то процесс завершится.\\ \\
\textbf{Алгоритм Евклида. Заключительные шаги.}
$$r_{k-2}(x) = r_{k-1}(x)h_k(x) + r_k(x)$$
$$r_{k-1}(x) = r_k(x)h_{k+1}(x)$$
$$(r_{k-2}(x), r_{k-1}(x)) = (r_{k-1}(x), r_{k}(x))$$
\indentСтрого говоря, $(r_{k-1}(x), r_{k}(x))$ необязательно равен $r_k(x)$. $r_k(x)$ является лишь одним из НОД.

\subsubsection{Каноническое разложение}
\begin{defn}
Пусть для многочлена $f(x)$ существует разложение: $$f(x) = ap_1(x)p_2(x)...p_k(x),$$ где все многочлены $p_i$ - неприводимые и нормированные. Тогда такое разложение называют \textbf{разложением на неприводимые множители} или \textbf{факторизацией многочлена}.
\end{defn}

\begin{defn}
Пусть для многочлена $f(x) \in F[x]$ существует разложение: $$f(x) = a_0(p_1(x))^{k_1}(p_2(x))^{k_2}...(p_r(x))^{k_r},$$ где все многочлены $p_i$ - неприводимые, нормированные и попарно различные. Тогда такое разложение называют \textbf{каноническим разложением над полем}, а значение $k_i$ - \textbf{кратностью множителя} $p_i$. Если $k_i = 1$, то множитель $p_i$ называется \textbf{простым}.
\end{defn}

~\
\begin{task*}
Дан многочлен $f$. Нужно найти вид $f(x) = a\varphi_1(x)(\varphi_2(x))^2...(\varphi_s(x))^s,$ в котором $\varphi_i$ - произведение всех множителей кратности $i$.
\end{task*}

\begin{examp}
Получить каноническое разложение многочлена $$f(x) = (x - 1)(x - 2)(x^2 + x + 1)^2(x^2 - x + 1)^2(x^3 - 2)^3.$$
$$f(x) = \varphi_1(x)(\varphi_2(x))^2(\varphi_3(x))^3.$$
$$\varphi_1(x) = (x - 1)(x - 2).$$
$$\varphi_2(x) = (x^2 + x + 1)(x^2 - x + 1).$$
$$\varphi_3(x) = x^3 - 2.$$
\end{examp}

\section{Лекция 2.}
\subsection{Каноническое разложение}
\hspace{0.6cm}Рассмотрим каноническое разложение многочлена $f(x)$:
$$f(x) = a_0(p_1(x))^{k_1}(p_2(x))^{k_2}...(p_r(x))^{k_r}$$
\indentВынесем первый полином $p_1(x)$:
$$f(x) = a_0(p_1(x))^{k_1}(p_2(x))^{k_2}...(p_r(x))^{k_r} = (p_1(x))^{k_1}g(x),~(g(x), p_1(x)) = 1.$$
$$f'(x) = k_1(p_1(x))^{k_1 - 1} \cdot (p_1(x))'g(x) + (p_1(x))^{k_1}g'(x) =$$
$$ = (p_1(x))^{k_1 - 1} \cdot (k_1(p_1(x))'g(x) + p_1(x)g'(x)).$$
\indentДокажем, что многочлен $k_1(p_1(x))'g(x)~+~p_1(x)g'(x)$ не делится на $p_1(x)$. Допустим, что он делится. Так как второе слагаемое $p_1(x)g'(x)$ делится на $p_1(x)$, то должно делиться и первое. Однако $(p_1(x))'$ не делится на $p_1(x)$, так как его степень меньше, чем у $p_1(x)$. Но и $(g(x), p_1(x)) = 1,$ следовательно, первое слагаемое не делится на $p_1$, не делится и вся сумма. Полученное противоречие доказывает, что многочлен $k_1(p_1(x))'g(x)~+~p_1(x)g'(x)$ не делится на $p_1(x)$.
\\
\indentТаким образом, если неприводимый многочлен $p(x)$ входит в каноническое разложение $f(x)$ в степени $k$,  то этот многочлен входит в каноническое
разложение $f'(x)$ в степени $k - 1$.

$$f'(x) = na_0(p_1(x))^{k_1 - 1}(p_2(x))^{k_2 - 1}~...~(p_r(x))^{k_r - 1}\underset{\text{эти многочлены есть, но неинтересны}}{(p_{r+1}(x))^{k_{r+1}}(p_{r+2}(x))^{k_{r+2}}~...}$$
$$(f(x), f'(x)) = (p_1(x))^{k_1 - 1}(p_2(x))^{k_2 - 1}~...~(p_r(x))^{k_r - 1}.$$

~\

\indent Будем предполагать, что старший коэффициент равен 1.
$$f(x) = \varphi_1(x) \cdot (\varphi_2(x))^2 \cdot ... \cdot {\varphi_k(x)}^k.$$
$$(f(x), f'(x)) = \varphi_2(x) \cdot (\varphi_3(x))^2 \cdot ... \cdot {\varphi_k(x)}^{k - 1}.$$

$$u_1(x) = f(x).$$
$$u_2(x) = (f(x), f'(x)) = (u_1(x), u'_1(x)).$$
$$u_3(x) = (u_3(x), u'_3(x)) = \varphi_3(x) \cdot (\varphi_4(x))^2 \cdot ... \cdot {\varphi_k(x)}^{k - 2}.$$
$$u_4(x) = (u_4(x), u'_4(x)) = \varphi_4(x) \cdot (\varphi_5(x))^2 \cdot ... \cdot {\varphi_k(x)}^{k - 3}.$$
$$...$$
$$u_{k - 1}(x) = (u_{k-2}(x), u'_{k-2}(x)) = \varphi_k(x).$$
$$u_{k}(x) = (u_{k-1}(x), u'_{k-1}(x)) = 1.$$

$$v_1(x) = \dfrac{u_1(x)}{u_2(x)} =  \varphi_1(x) \cdot \varphi_2(x) \cdot ... \cdot \varphi_k(x).$$
$$v_2(x) = \dfrac{u_2(x)}{u_3(x)} =  \varphi_2(x) \cdot ... \cdot \varphi_k(x).$$
$$...$$
$$v_{k-1}(x) = \dfrac{u_{k-1}(x)}{u_k(x)} = \varphi_k(x).$$

$$\varphi_1(x) = \dfrac{v_1(x)}{v_2(x)},~\varphi_2(x) = \dfrac{v_2(x)}{v_3(x)},~...$$

\subsection{Уравнения третьей степени}
\subsubsection{Уравнения с комплексными коэффициентами}
$$a_0x^3 + a_1x^2 + a_2x + a_3 = 0, a_i \in \mathbb{C}, a_0 \neq 0.$$

\noindent\textbf{Шаг 1.} Разделим обе части уравнения на $a_0$.
$$x^3 + ax^2 + bx + c = 0.$$

\noindent\textbf{Шаг 2.} Введем замену $x = y - \dfrac{a}{3}$.
$$(y - \dfrac{a}{3})^3 + a(y - \dfrac{a}{3})^2 + b(y - \dfrac{a}{3}) + c = 0.$$
$$y^3 - \stkout{ay^2} + ... + \stkout{ay^2} + ... = 0 \text{ (других квадратов нет)}.$$

Получено уравнение вида $x^3 + px + q = 0,~p, q \in \mathbb{C}$.
Рассмотрим простейшее уравнение третьей степени $x^3 = 1$:
$$x^3 = 1 \Rightarrow x = cos \frac{2 \pi k}{3} + i~sin \frac{2 \pi k}{3}, k = 0,1,2.$$
\begin{itemize}
\item $k = 0 \Rightarrow x =  cos(0) + i \cdot sin(0)= 1 + 0 = 1$.
\item $k = 1 \Rightarrow x = cos(\dfrac{2\pi}{3}) + i \cdot sin(\dfrac{2\pi}{3})= -\dfrac{1}{2} + i\dfrac{\sqrt{3}}{2} = \omega.$
\item $k = 2 \Rightarrow x = cos(\dfrac{4\pi}{3}) + i \cdot sin(\dfrac{4\pi}{3}) = -\dfrac{1}{2} - i\dfrac{\sqrt{3}}{2} = \omega^2.$
\end{itemize}

Рассмотрим общий случай: $x^3 = a,~a \in \mathbb{C},~a \neq 0$, если есть корень $x_0$, то: $$x_0 = x_0\cdot1,~x_1 = x_0 \omega,~x_2 = x_0 \omega^2.$$

Теперь переменную $x$ рассмотрим как сумму переменных $u$ и $v$: $x = u + v$.
$$(u + v)^3 + p(u + v) + q = 0.$$
$$u^3 + 3u^2v + 3uv^2 + v^3 + p(u + v) + q = 0.$$
$$(u^3 + v^3 + q) + (u + v)(3uv + p) = 0.$$
\indentПотребуем, чтобы $u^3 + v^3 + q = 0$ и $3uv + p = 0$.

\begin{align*}
\begin{cases}
   u^3 + v^3 + q = 0, \\
   3uv + p = 0.
\end{cases}
&
\Rightarrow
&
 \begin{cases}
   u^3 + v^3 = -q, 
   \\
   uv = -\dfrac{p}{3}.
 \end{cases}
\end{align*}
\indentВыполним (неэквивалентный!) переход к $u^3v^3$.
\begin{equation*}
 \begin{cases}
   u^3 + v^3 = -q, 
   \\
   u^3v^3 = -\dfrac{p^3}{27}.
 \end{cases}
\end{equation*}

Так как переход к кубу неэквивалентен, то появятся лишние решения, поэтому нужно будет вернуться к уравнению: $uv = -\dfrac{p}{3}.$ \newline
\indent
Значения $u^3$ и $v^3$ можно рассматривать в качестве корней следующего квадратного уравнения:
$$z^2 + qz -\dfrac{p^3}{27} = 0.$$
$$z = -\frac{q}{2} \pm \sqrt{\frac{q^2}{4} + \frac{p^3}{27}}.$$
$$x = \sqrt[3]{-\frac{q}{2} + \sqrt{\frac{q^2}{4} + \frac{p^3}{27}}} + \sqrt[3]{-\frac{q}{2} - \sqrt{\frac{q^2}{4} + \frac{p^3}{27}}} \textbf{  (Формула Кардано)}.$$

~\

$$u^3v^3 = -\dfrac{p^3}{27}.$$
$$uv = -\dfrac{p}{3} \text{ или }  -\dfrac{p}{3}\omega \text{ или } -\dfrac{p}{3}\omega^2.$$

Пусть найдены $u_0, v_0 \Rightarrow u_0v_0 = -\dfrac{p}{3}$.
$$u_1 = u_0\omega, v_1 = v_0\omega^2$$
$$u_2 = u_0\omega^2, v_2 = v_0\omega$$
$$x_1 = u_0 + v_0$$
$$x_2 = \omega u_0 + \omega^2 v_0$$
$$x_3 = \omega^2 u_0 + \omega v_0$$
$$x_2 = -\frac{u_0 + v_0}{2} + i\sqrt{3} \ \frac{u_0 - v_0}{2}$$
$$x_3 = -\frac{u_0 + v_0}{2} - i\sqrt{3} \ \frac{u_0 - v_0}{2}$$

\subsubsection{Уравнения с рациональными коэффициентами}
\begin{defn}
Рассмотрим следующее уравнение: $$x^3 + px + q = 0,~p,q \in \mathbb{R},~p \neq 0.$$
\textbf{Дискриминантом} такого уравнения называют выражение $D$: $$D = -108(\dfrac{q^2}{4} + \dfrac{p^3}{27}) = -27q^2 - 4p^3.$$
\end{defn}

\begin{defn}
Рассмотрим следующее уравнение: $$x^n + a_1x^{n-1} + ... + a_n = 0,$$
у которого есть корни $x_1, x_2, ..., x_n$.
\textbf{Дискриминантом} такого уравнения называют выражение $D$: $$D = \prod_{1 \leq i < j \leq n }(x_i - x_j)^2.$$
\end{defn}

~\

$\pmb{D > 0.}$
$$\dfrac{q^2}{4} + \dfrac{p^3}{27} < 0 \Rightarrow p < 0, uv = -\dfrac{p}{3} > 0.$$
$$x = \sqrt[3]{A + Bi} + \sqrt[3]{A - Bi}.$$
$$|A + Bi| = |A - Bi|.$$
$$u = R \cdot (cos(\varphi) + i \cdot sin(\varphi)).$$
$$v = R \cdot (cos(\psi) + i \cdot sin(\psi)).$$
$$R = \sqrt[6]{A^2 + B^2}.$$
$$uv = R^2 \cdot (cos(\varphi + \psi) + i \cdot sin(\varphi + \psi)).$$
$$\varphi = -\psi \Rightarrow u + v = 2R \cdot cos(\varphi).$$
$$u - v = 2i \cdot sin(\varphi).$$
$$x_1 = u + v \in \mathbb{R}.$$
$$x_{2,3} = -\dfrac{u + v}{2} \pm i \sqrt{3} \cdot \dfrac{2i \cdot sin(\varphi)}{2} \in \mathbb{R}.$$

~\

$\pmb{D < 0.}$
$$x = \sqrt[3]{A + B}  + \sqrt[3]{A - B}.$$
$$B \neq 0 \Rightarrow A + B \neq A - B \Rightarrow \sqrt[3]{A + B} \neq \sqrt[3]{A - B}.$$
$$u = \sqrt[3]{A + B} \in \mathbb{R}.$$
$$x_1 = u + v \in \mathbb{R}.$$
$$x_{2, 3} = -\dfrac{u + v}{2} \pm \dfrac{i\sqrt{3}}{2}(u - v) \in \mathbb{C}.$$

~\

$\pmb{D = 0.}$
$$x = \sqrt[3]{A} + \sqrt[3]{A}.$$
\indent$u$ - вещественный кубический корень из $A$, $uv = -\dfrac{p}{3} \Rightarrow v \in \mathbb{R}$, но $v$ - тоже вещественный кубический корень из $A$, следовательно, $u = v$.
$$\begin{cases}
x_1 = u + v = 2u, \\
x_{2,3} = -\dfrac{u + v}{2} \pm \dfrac{i\sqrt{3}}{2}(u - v) = -u.
\end{cases}$$

\section{Лекция 3.}
\subsection{Решение уравнений четвертой степени}
$$x^4 + ax^3 + bx^2 + cx + d = 0$$
$$(x^2 + \frac{a}{2}x + y)^2 = x^4 + ax^3 + \frac{a^2}{4}x^2 + 2x^2y + axy + y^2$$
$$\stkout{x^4} + \stkout{ax^3} + bx^2 + cx + d = \stkout{x^4} + \stkout{ax^3} + \frac{a^2}{4}x^2 + 2x^2y + axy + y^2$$
$$(x^2 + \frac{a}{2}x + y)^2 = Ax^2 + Bx + C,\, A = \frac{a^2}{4} + 2y - b, B = ay - c, C = y^2 - d.$$
\indent Необходимо подобрать $y$ так, чтобы справа был полный квадрат. Для этого необходимо, чтобы \textbf{резольвента Феррари} $B^2 - 4AC$ была равна нулю.
$$(x^2 + \frac{a}{2}\,x + Y)^2 = (\sqrt{A}\,x + \sqrt{C})^2$$
$$x^2 + \frac{a}{2}\,x + Y = \pm(\sqrt{A}\,x + \sqrt{C})$$
\begin{examp}
$x^4 - 2x^3 + 2x^2 + 4x - 8 = 0.$
$$(x^2 - x + y)^2 = x^4 - 2x^3 + 2x^2y + x^2 - 2xy + y^2$$
$$\stkout{x^4} - \stkout{2x^3} + 2x^2 + 4x - 8 + (x^2 - x + y)^2 = \stkout{x^4} - \stkout{2x^3} + 2x^2y + x^2 - 2xy + y^2$$
$$(x^2 - x + y)^2  = (2y - 1)x^2 - (2y + 4)x + (y^2 + 8)$$
$$(2y + 4)^2 - 4(2y - 1)(y^2 + 8) = 0$$
$$4y^2 + 16y + 16 - 8y^3 - 64y + 4y^2 + 32 = 0$$
$$y^3 - y^2 + 6y - 6 = 0 \Rightarrow y_1 = 1 \textup{ (найден с помощью подбора)}$$
$$(x^2 - x + 1)^2 = x^2 - 6x + 9$$
$$(x^2 - x + 1)^2 = (x - 3)^2$$
\\
$$x^2 - x + 1 = x - 3$$ 
$$x^2 - 2x + 4 = 0$$
$$x_{1,2} = 1 \pm \sqrt{-3} = 1 \pm \sqrt{3}i$$
\\
$$x^2 - x + 1 = 3 - x$$
$$x^2 - 2  = 0$$
$$x_{3,4} = \pm \sqrt{2}$$
\end{examp}

\subsection{Границы комплексных корней}
\begin{thm}[\textbf{Теорема о границах комплексных корней}]
Рассмотрим полином $f(x) = a_0x^n + a_1x^{n-1} + ... + a_n.$ Введем величину $A = \max\limits_{1 \leq k \leq n} |a_n|$.
Все комплексные корни этого многочлена удовлетворяют неравенству: $$|x| < 1 + \frac{A}{|a_0|}$$.
\end{thm}
\begin{proof}[\textbf{Доказательство}]
Докажем, что если $|x| \geq 1 + \dfrac{A}{|a_0|}$, то $f(x) \neq 0$. \\
\indent Воспользуемся стандартными неравенствами:
\begin{enumerate}
\item $|z_1 + z_2| \geq |z_1| - |z_2|$
\item $|\sum\limits_{k=1}^n z_k| \leq \sum\limits_{k=1}^n |z_k|$
\end{enumerate}
$$|f(x)| = |a_0x^n + a_1x^{n-1} + ... + a_n| \geq |a_0x^n| - |a_1x^{n-1} + a_2x^{n-2} + ... + a_n|$$
$$|a_1x^{n-1}| = |a_1||x|^{n-1} \leq A|x|^{n-1}$$
$$|f(x)| \geq |a_0||x^n| - (|a_1x^{n-1}| + |a_2x^{n-2}| + ... + |a_n|)$$
$$|f(x)| \geq |a_0||x^n| - A(|x|^{n-1} + |x|^{n-2} + ... + 1) = |a_0||x^n| - A\frac{|x|^n - 1}{|x| - 1} =$$
$$= |a_0||x^n| - A\frac{|x|^n}{|x| - 1} + \underbrace{A\frac{1}{|x| - 1}}_{>0} > |a_0||x^n| - A\frac{|x|^n}{|x| - 1} =$$
$$= \frac{|x|^n|a_0|}{|x| - 1}\underbrace{(|x| - 1 - \frac{A}{|a_0|})}_{\text{по условию }\geq 0} \geq 0.$$
\indent Следовательно, $|f(x)| > 0$.
\end{proof}

\subsection{Границы положительных корней}
\subsubsection{Метод Маклорена}
\indent \indent Рассмотрим многочлен с коэффициентами из $\mathbb{R}: f(x) = a_0x^n + a_1x^{n-1} + ... + a_n$. \\
Предполагаем, что $a_0 > 0$. Если все числа больше нуля, то положительных корней нет. Поэтому считаем, что $\exists k: 1 \leq k  \leq n, a_k < 0.$ 
\begin{thm}[\textbf{Оценка по методу Маклорена}]
Пусть $k~-$ номер первого отрицательного коэффициента многочлена, $B~-$ наибольший из модулей отрицательных коэффициентов. Тогда все положительные корни многочлена удовлетворяют неравенству:
$$x < 1 + \sqrt[k]{\frac{B}{a_0}}$$
\end{thm}
\begin{proof}[\textbf{Доказательство}]
Докажем, что если $x \geq 1 + \sqrt[k]{\dfrac{B}{a_0}}$, то $f(x) > 0$. \\
$$f(x) =  a_0x^n + \underbrace{a_1x^{n-1} + ... + a_{k-1}x^{n-k+1}}_{\geq 0} + a_kx^{n-k} + a_{k+1}x^{n-k-1} + ... + a_n$$
$$|a_k| \leq B \Rightarrow -B \leq a_k \leq B$$
\indent Если $a_{k+1} \geq 0,$ тогда $a_{k+1} > -B$, так как $-B$ отрицательное число. \\
\indent Если $a_{k+1} < 0,$ тогда $a_{k+1} \geq -B$. \\
\indent В любом случае $a_{k+1} \geq -B$, аналогично для следующих коэффициентов.
$$f(x) \geq a_0x^n - B(x^{n-k} + x^{n-k-1} + ... + 1) = a_0x^n - B\frac{x^{n-k+1} - 1}{x - 1} =$$
$$= a_0x^n - B\frac{x^{n-k+1}}{x - 1} + \underbrace{B\frac{1}{x - 1}}_{>0} > a_0x^n - B\frac{x^{n-k+1}}{x - 1}
= \frac{a_0x^{n-k+1}}{x-1}((x-1)x^{k-1} - \frac{B}{a_0})$$
$$x^{k-1} \geq (x-1)^{k-1}$$
$$x > 1 + \sqrt[k]{\dfrac{B}{a_0}} \Rightarrow (x-1)^k \geq \frac{B}{a_0}$$
$$f(x) > \frac{a_0}{x-1}x^{n-k+1}\underbrace{((x-1)^k - \frac{B}{a_0})}_{\geq 0} \geq 0$$
\indent Следовательно, $|f(x)| > 0$.
\end{proof}
\subsubsection{Метод Ньютона}
\begin{thm}[\textbf{Оценка по методу Ньютона}]
Рассмотрим полином: $$f(x) = a_0x^n + a_1x^{n-1} + ... + a_n, a_i \in \mathbb{R}, a_0 > 0.$$ Предположим, что: $$\exists c: f(c) > 0, f'(c) > 0, f''(c) > 0, ..., f^{(n)}(c) > 0.$$ Тогда все положительные корни многочлена удовлетворяют неравенству: $x < c$.
\end{thm}
\begin{proof}[\textbf{Доказательство}]
$$f(x) = f(c) + \frac{f'(c)}{1!}(x-c) + \frac{f''(c)}{2!}(x-c)^2 + ... + \frac{f^{(n)}(c)}{n!}(x-c)^n \text{ (формула Тейлора)}$$
\indent Если $x \geq c$, то справа будет стоять строго положительное число, то есть не может быть корнем для $f(x)$.
\end{proof}

\subsubsection{Схема Горнера}
$$f(x) = (x - \alpha)g(x) + r(x)$$
$\begin{array}{c|c|c|c|c|c}
        & a_0 & a_1 & a_2 & ... & a_n \\ \hline
      \alpha  & a_0 & a_1 - \alpha a_0  & a_2 - \alpha a_1 & ... & a_n - \alpha a_{n-1} \\ 
\end{array}$
$$f(x) = a_0x^n + a_1x^{n-1} + ... + a_n$$
$$g(x) = b_0x^{n-1} + b_1x^{n-2} + ...  + b_{n-1}$$
$$a_0x^n + a_1x^{n-1} + ... + a_n = (b_0x^{n-1} + b_1x^{n-2} + ...  + b_{n-1})(x - \alpha) + r$$
$a_0 = b_0, a_1 = -b_0\alpha + b_1, a_2 = b_2 - \alpha b_1, ..., a_n = r - \alpha b_{n-1}$ \\
$b_0 = a_0, b_1 = a_1 + \alpha b_0, b_2 = a_2 + \alpha b_1, ..., b_{n-1} = a_{n-1} + \alpha b_{n-2}, r= a_n + \alpha b_{n-1}$
\\
$$f(x) = g(x)(x - \alpha) + r$$
$$g(x) = h(x)(x - \alpha) + r_1$$
$$f(x) = h(x)(x - \alpha)^2 + r1(x-\alpha) + r$$
$$f(x) = \varphi(x)(x-\alpha)^3 + r_2(x-\alpha)^2 + r_1(x-\alpha) + r$$
\indent Так как разложение единственное, то $r = f(a), r_1 = f'(a), r_2 = \dfrac{f''(a)}{2!}, ...$ \\
\indent Если все остатки положительные, то $\alpha~-$ верхняя граница по методу Ньютона.

\begin{examp}
$f(x) = 3x^4  - 18x^3 + 14x^2 + 36x + 25$
$$a_0 = 3, B = 18, k = 1,  1 + \sqrt[k]{\dfrac{B}{a_0}} = 1 + \frac{18}{3} = 7$$
\indent Оценка по Маклорену: положительные корни меньше 7. \\

$\begin{array}{c|c|c|c|c|c}
        & 3 & -18 & 14 & 36 & 25 \\ \hline
      4  & 3 & -6  & -10 & -4 & \textbf{9} \\  \hline
      4 & 3 & 6 & 14 & \textbf{52}
\end{array}$ \\ \\
\indent Далее вычисления можно не продолжать, так все коэффициенты положительные, а значит, значения всех далее взятых производных при положительных аргументах будут тоже положительными, то есть выполняется условие метода Ньютона. Это означает, что выше 4 положительные корни искать не нужно.
\end{examp}

Нижняя граница отрицательных корней находится с помощью многочлена $g(x) = f(-x)$. Если для $g$ выполняется $x \leq M$, то для $f$ выполняется $x \geq -M$. Нижняя граница положительных корней находится c помощью $h(x) = x^nf(\dfrac{1}{x})$. Если для $h$ выполняется $x \leq M$, то для $f$ выполняется $x \geq \dfrac{1}{M}$.

\section{Лекция 4.}
\subsection{Теорема Штурма}
\begin{defn}
Пусть $f(x)$ - многочлен с вещественными коэффициентами. Рассмотрим последовательность многочленов: $$f_0(x) = f(x), f_1(x), f_2(x), ..., f_k(x).$$ Многочлены $f_1, ..., f_{k-1}$ будем называть промежуточными. Данная последовательность многочленов называется \textbf{рядом Штурма} для многочлена $f$ на отрезке $[a,b]$, если числа $a$ и $b$ не являются корнями многочлена $f$ и выполняются следующие условия:
\begin{enumerate}
\item{Соседние многочлены ряда не имеют общих корней на $[a,b]$;}
\item {Последний многочлен ряда не имеет корней на $[a,b]$;}
\item {Если $\alpha \in [a,b]$ - корень промежуточного многочлена, то соседние с ним многочлены ряда Штурма имеют в этой точке значения разных знаков;}
\item {При прохождении корня $\alpha \in [a,b]$ многочлена $f$ произведение $f_0 f_1$ меняет знак с минуса на плюс.}
\end{enumerate}
\end{defn}

\begin{defn}
Пусть есть последовательность $a_1, a_2, ..., a_n, \forall a_i \in \mathbb{R}.$ Заменим плюсом, минусом или нулем элементы в зависимости от значения, затем удалим нули из последовательности. Любая пара плюса и минуса, находящихся рядом, называется \textbf{переменой знака} этой последовательности.
\end{defn}
\begin{examp}
$$3,6,-2,0,5,4,-9,3,2$$
$$+,+,-,0,+,+,-,+,+$$
$$+,(+,[-),+],(+,[-),+],+ $$
$$\text{   (скобками выделены перемены знака)}$$ 
\end{examp}

\begin{thm}
Пусть $f_0, f_1, ..., f_k$ - ряд Штурма для многочлена $f$ на $[a,b]$. Тогда обозначим через $W(x)$, где $x \in [a,b]$, число перемен знака в последовательности $f_0(a), f_1(a), ..., f_k(a)$. Тогда число корней многочлена $f$ на $[a,b]$ равно: $W(a) - W(b)$.
\end{thm}
\begin{proof}[\textbf{Доказательство}]
Многочлен является непрерывной функцией, поэтому пока мы не пройдем корень многочлена, то число перемен знака не поменяется. Покажем, что если пройден корень промежуточного многочлена, то число перемен знака остается прежним. \\
\\
\indent Пусть $\alpha$ - корень промежуточного многочлена $f_r$, всегда можно выбрать число $\varepsilon$, так, что в $[\alpha - \varepsilon,  \alpha + \varepsilon]$ у многочленов $f_{r-1}$ и $f_{r+1}$ не будет корней, то есть на этом участке они сохраняют свой знак. Тогда в соответствии с условиями ряда Штурма получаем: \\
$$\begin{array}{c|c|c|c|c}
        & f_{r-1} & f_r & f_{r+1} & \textit{ЧПЗ} \\ \hline
      x = \alpha - \varepsilon & - & \mp  & + & 1 \\  \hline
      x = \alpha & - & 0  & + & 1 \\  \hline
      x = \alpha + \varepsilon & - & \pm  & + & 1
\end{array}
~~~~~
\begin{array}{c|c|c|c|c}
        & f_{r-1} & f_r & f_{r+1} & \textit{ЧПЗ} \\ \hline
      x = \alpha - \varepsilon & + & \mp  & - & 1 \\  \hline
      x = \alpha & + & 0  & - & 1 \\  \hline
      x = \alpha + \varepsilon & + & \pm  & - & 1
\end{array}
$$
\indent Число перемен знака не изменилось. Многочлен $f_k(x)$ не имеет корней, осталось рассмотреть многочлен $f_0(x)$.
$$\begin{array}{c|c|c|c}
        & f_0 & f_1 & \textit{ЧПЗ} \\ \hline
      x = \alpha - \varepsilon & - & + & 1 \\  \hline
      x = \alpha & 0  & + & 0 \\  \hline
      x = \alpha + \varepsilon & + & + & 0
\end{array}
~~~~~
\begin{array}{c|c|c|c}
        & f_0 & f_1 & \textit{ЧПЗ} \\ \hline
      x = \alpha - \varepsilon & + & - & 1 \\  \hline
      x = \alpha & 0  & - & 0 \\  \hline
      x = \alpha + \varepsilon & - & - & 0
\end{array}$$
\indent При прохождения корня многочлена $f_0$ число перемен знака уменьшается. Но корень $f_0$ является корнем многочлена $f$.
\end{proof}

\subsection{Построение ряда Штурма}
\indent \indent Пусть $f(x)$ не имеет кратных корней.
$$f_0(x) = f(x)$$
$$f_1(x) = f'(x)$$
$$f_{k+2} = \text{остаток от }\frac{f_k(x)}{f_{k+1}(x)}, \text{ взятый с противоположным знаком}$$
\indent Если многочлены не имеют кратных корней, следовательно, они взаимно простые со своими производными.
$$f_0(x) = f_1(x)g_1(x) - f_2(x)$$
$$f_1(x) = f_2(x)g_2(x) - f_3(x)$$
$$...$$
\indent Докажем, что это ряд Штурма.
\begin{enumerate}
\item Пусть $f_2(\alpha) = 0$ и $f_3(\alpha) = 0$. Тогда $f_1(\alpha) = 0$ и $f_0(\alpha) = 0$, $f(\alpha) = 0$, $f'(\alpha) = 0$.
То есть $\alpha$ является кратным корнем многочлена. Полученное противоречие доказывает, что соседние многочлены не имеют общих корней.
\item Пусть $f_2(\alpha) = 0$. Тогда $f_1(\alpha) = -f_3(\alpha)$. Так соседние многочлены $f_1, f_3$ не равны в точке $\alpha$ нулю, то они имеют в этой точке значения разных знаков.
\item Последний многочлен не имеет корней, так как является константой.
\item Пусть $f_0(\alpha) = f(\alpha) = 0$. Тогда $f'(\alpha) \neq 0$. Если $f'(\alpha) > 0$, то $f$ возрастает и $f(x)$ меняет знак с - на +, произведение $f_0f_1$ меняет знак с - на +. Если $f'(\alpha) < 0$, то $f$ убывает и $f(x)$ меняет знак с + на -, произведение $f_0f_1$ меняет знак с - на +.
\end{enumerate}
\indent Эти свойства выполняются для любых отрезков, кроме случаев, когда концы отрезков являются корнями многочлена.
\begin{examp}
Построить ряд Штурма и отделить корни многочлена (найти промежутки, на которых многочлен имеет единственный корень).
$$f_0(x) = x^4 - 12x^2 - 16x - 4$$
$$f_1(x) = 4x^3 - 24x - 16 = x^3 - 6x - 4$$
\polylongdiv[style=B]{x^4 - 12x^2 - 16x - 4}{x^3 - 6x - 4}
\\
$$f_2(x) = -(-6x^2 - 12x - 4) = 3x^2 + 6x + 2$$
\polylongdiv[style=B]{3x^3 - 18x - 12}{3x^2 + 6x + 2}
\\
$$f_3(x) = x + 1$$
\polylongdiv[style=B]{3x^2 + 6x + 2}{x + 1}
\\
$$f_4(x) = 1$$
$\begin{array}{c|c|c}
               & -\infty & \infty \\ \hline
       f_0  & + & + \\
       f_1 & - & + \\
       f_2 & + & + \\
       f_3 & - & + \\
       f_4 & + & + \\ \hline
     \textit{ЧПЗ} & 4 & 0
\end{array}$
Таким образом, многочлен имеет 4 вещественных корня. \\ \\
\indent Отделим корни по методу Маклорена: $$a_0 = 1, k = 2, B = 16, 1+ \sqrt{\dfrac{16}{1}} = 5.$$
$$g(x) = f(-x) = x^4 - 12x^2 + 16x - 4, a_0 = 1, k = 2, B = 12, 1 + \sqrt{\dfrac{12}{1}} = 1 + 2\sqrt{3}.$$
Отрицательные корни больше $-1 - 2\sqrt{3}$ (то есть больше -5). Ищем корни в (-5; 5). \\ \\
$\begin{array}{c|c|c|c|c|c|c|c|c|c|c|c}
               & -5 & -4 & -3 & -2 & -1 & 0 & 1 & 2 & 3 & 4 & 5 \\ \hline
       f_0 & + & + & + & - & + & - & - & - & - & - & +\\
       f_1 & - & - & - & 0 & + & - & - & - & + & + & +\\ 
       f_2 & + & +  & + & + & - & + & + & + & + & +  & + \\
       f_3 & - & -  & - & - & 0 & + & + & + & + & + & + \\
       f_4 & + & + & + & + & + & + & +  & + & + & + & +\\ \hline
      \textit{ЧПЗ} & 4 & 4 & 4 & 3 & 2 & 1 & 1 & 1 & 1 & 1 & 0
\end{array}$ \\ \\
Ответ:
\begin{itemize}
\item[1)] $f_0(x) = x^4 - 12x^2 - 16x - 4$
\item[2)] $f_1(x) = x^3 - 6x - 4$
\item[3)] $f_2(x) = 3x^2 + 6x + 2$
\item[4)] $f_3(x) = x + 1$
\item[5)] $f_4(x) = 1$
\end{itemize}
\indent  \indent Многочлен $f$ имеет по одному вещественному корню на промежутках (-3; -2), (-2; -1), (-1; 0) и (4; 5).
\end{examp}
\subsection{Примеры нестандартного построения ряда Штурма}
$$f_0(x) = f_n(x) = 1 + x + \frac{x^2}{2!} + \frac{x^3}{3!} + ... + \frac{x^n}{n!}$$
$$f_1(x) = f'_n(x) = 1 + x + \frac{x^2}{2!} + \frac{x^3}{3!} + ... + \frac{x^{n-1}}{(n-1)!} = f_{n-1}(x)$$
$$f_2(x) = -f_n(x) + f'_n(x) = -\frac{x^n}{n!}$$
\indent Неотрицательных корней этот многочлен не имеет, поэтому ищем корни на $(-\infty; -\varepsilon)$, где $\varepsilon > 0$ и малое.

\begin{enumerate}
\item Последний многочлен не имеет корней на $(-\infty; -\varepsilon)$, так как его единственный корень равен нулю.
\item Так как $f_2$ не имеет корней на $(-\infty; -\varepsilon)$, то и общих корней с $f_0$ на $(-\infty; -\varepsilon)$ нет.
\item Пусть $f_1(\alpha) = 0$. $f_0 - f_1 = -f_2, f_0(\alpha) = -f_2(\alpha)$, так как $f_2(\alpha) \neq 0$, то они имеют разный знак.
\item  Пусть $f(\alpha) = 0$. Тогда $f'(\alpha) \neq 0$. \\ Если $f'(\alpha) > 0$, то $f$ возрастает и $f(x)$ меняет знак с - на +, произведение $f_0f_1$ меняет знак с - на +. \\ Если $f'(\alpha) < 0$, то $f$ убывает и $f(x)$ меняет знак с + на -, произведение $f_0f_1$ меняет знак с - на +.
\end{enumerate}
\indent Условия выполнены, это ряд Штурма. \\ \\
$\begin{array}{c|c|c|c|c}
  & f_0 & f_1 & f_2 & \textit{ЧПЗ} \\ \hline
x = -\infty & (-1)^n & (-1)^{n-1} & (-1)^{n+1} & 1 \\ \hline
x = -\varepsilon & + & + & (-1)^{n+1} & \begin{cases}
0, & \text{если $n$ - нечетное} \\
1, & \text{если $n$ - четное}
\end{cases}
\end{array}$
\\ \\
\indent Если $n$ - четное, то корней нет, если $n$ - нечетное, то имеет 1 корень. \\
\indent Проблема: если $f$ имеет кратные корни?
$$f(x) - f'(x) = \frac{x^n}{n!}$$
\indent Правая часть имеет среди делителей только степени $x$, но левая часть не делится на $x$, поэтому кратных корней нет.
\begin{examp}
$$f(x) = f_0(x) = 2x^3 - 3x^2 + 6x + 11$$
$$f'(x) = 6x^2 - 6x + 6$$
$$f_1(x) = x^2 - x + 1  \text{ не имеет корней в }\mathbb{R}.$$
\indent Все условия выполняются, $f_0, f_1$ являются рядом Штурма. \\
$$\begin{array}{c|c|c|c}
  & f_0 & f_1 & \textit{ЧПЗ} \\ \hline
x = -\infty & - & + & 1\\ \hline
x = \infty & + & + & 0
\end{array}
~~~~\text{Только один вещественный корень.}$$
\indent Ряд Штурма можно обрывать, если текущий многочлен не имеет вещественных корней.
\end{examp}

\begin{examp}
Выразить через элементарные симметрические многочлены многочлен $f = x_1^4 + x_2^4 + x_3^4 - 2x_1^2x_2^2 - 2x_1^2x_3^2 - 2x_2^2x_3^2$. \\ \\
\indent Старший член - $x_1^4$. \\ \\
$\begin{array}{c|c|c}
  k_1 & k_2 & k_3 \\ \hline
  4 & 0 & 0 \\
  3 & 1 & 0 \\
  2 & 2 & 0 \\
  2 & 1 & 1
\end{array}$ \\ \\
$$f = \sigma_1^{4-0}\sigma_2^{0-0}\sigma_3^{0} + A\sigma_1^{3-1}\sigma_2^{1-0}\sigma_3^{0} + B\sigma_1^{2-2}\sigma_2^{2-0}\sigma_3^{0} + C\sigma_1^{2-1}\sigma_2^{1-1}\sigma_3^{1} =$$
$$ = \sigma_1^{4} + A\sigma_1^{2}\sigma_2 + B\sigma_2^{2} + C\sigma_1\sigma_3.$$
$$x_1 = 1, x_2 = 1, x_3 = 0 \Rightarrow f(x_1,x_2,x_3) = 0,~\sigma_1 = x_1 + x_2 + x_3 = 2,$$
$$\sigma_2 = x_1x_2 + x_1x_3 + x_2x_3 = 1,~\sigma_3 = x_1x_2x_3 = 0.$$
$$0 = 16 + 4A + B.$$
$$x_1 = 2, x_2 = 1, x_3 = 0 \Rightarrow f(x_1,x_2,x_3) = 9,~\sigma_1 = x_1 + x_2 + x_3 = 3,$$
$$\sigma_2 = x_1x_2 + x_1x_3 + x_2x_3 = 2, \sigma_3 = x_1x_2x_3 = 0.$$
$$9 = 81 + 18A + 4B.$$
\begin{align*}
\begin{cases}
  4A + B = -16\\
  18A + 4B = -72
\end{cases}
&
\Rightarrow
&
 \begin{cases}
   4A + B = -16\\
  9A + 2B = -36
 \end{cases}
&
\Rightarrow
&
 \begin{cases}
   8A + 2B = -32\\
  9A + 2B = -36
 \end{cases}
 &
\Rightarrow
 &
  A = -4, B = 0.
\end{align*}
$$f = \sigma_1^{4} - 4\sigma_1^{2}\sigma_2 + C\sigma_1\sigma_3.$$
$$x_1 = 1, x_2 = 1, x_3 = 1 \Rightarrow f(x_1,x_2,x_3) = -3, \sigma_1 = x_1 + x_2 + x_3 = 3,$$
$$\sigma_2 = x_1x_2 + x_1x_3 + x_2x_3 = 3, \sigma_3 = x_1x_2x_3 = 1.$$
$$-3 = 81 - 4 \cdot 27 + 3C \Rightarrow C = 8.$$
Ответ: $f = \sigma_1^4 - 4\sigma_1^2\sigma_2 + 8\sigma_1\sigma_3$.
\end{examp}

\section{Лекция 5.}
\subsection{Многочлены от нескольких переменных, определения}
\indent Будем рассматривать многочлены от переменных $x_1, x_2, ..., x_n$, где $n \geq 2$.
\begin{defn}
\textbf{Одночленами (мономами)} называются выражения вида $ax_1^{k_1}x_2^{k_2}...x_n^{k_n}$, где все коэффициенты $k_i$ - целые и неотрицательные.
\end{defn}
\begin{defn}
Число $\sum\limits_{i=1}^n k_i$ называют \textbf{полной степенью одночлена}.
\end{defn}
\begin{defn}
\textbf{Многочленом} называют сумму конечного числа одночленов.
\end{defn}

\subsection{Упорядочивание множества мономов}
\subsubsection{Отношение $\prec$}
\indent \indent При упорядочивании не обращают внимание на коэффициент $a$, кроме случая, если $a = 0$. \\
\indent Множество ненулевых мономов упорядочено с помощью отношения $\prec$, если:
\begin{itemize}
\item $\forall a,b,c: a \prec b, b \prec c \Rightarrow a \prec c.$
\item $\forall a,b \text{ выполняется только одно из соотношений: }a \prec b, a = b, b \prec a.$
\end{itemize}
$$\text{и}$$
\begin{itemize}
\item[1.] $a \prec b, \forall c: ac \prec bc.$
\item[2.] $a \neq 1 \Rightarrow 1 \prec a.$
\item[$2^{'}\!.$] $b \neq 1, \forall a: a \prec ab.$
\end{itemize}
\indent Условия $\bm{1,2}$ равносильны условиям $\bm{1,2'}$.
\begin{proof}[\textbf{Доказательство}]
\indent Пусть выполняется 1,2 и $b \neq 1$. Тогда $1 \prec b \Rightarrow a \prec ab$. \\
\indent Пусть выполняется $1,\!2^{'}$ и $b \neq 1$. Тогда $1 \prec b$. 
\end{proof}

\subsubsection{Способы упорядочивания}
\begin{enumerate}
\item \textbf{Лексикографическое}
$$x_1^{k_1}x_2^{k_2}...x_n^{k_n} \succ x_1^{l_1}x_2^{l_2}...x_n^{l_n}, \text{ если } k_1 = l_1, ..., k_t = l_t, k_{t+1} > l_{t+1}, t \geq 0.$$
Вводится термин: один моном старше или выше другого.
\item \textbf{Степенно-лексикографическое} \\
Либо $\sum\limits_{i=1}^n k_i$ > $\sum\limits_{i=1}^n l_i$, либо $\sum\limits_{i=1}^n k_i$ = $\sum\limits_{i=1}^n l_i$ и $\exists t \geq 0:  k_1 = l_1, ..., k_t = l_t, k_{t+1} > l_{t+1}.$
\end{enumerate}

\subsection{Старший член многочлена}
\indent \indent Пусть $f(x_1, ..., x_n)$ - ненулевой многочлен. Тогда существует член, который старше всех членов - старший или высший член.
$$x_1^2x_2x_3 \succ x_1x_2^3x_3^4$$
\begin{lem}[\textbf{Лемма о старшем члене}]
Старший член произведения многочленов равен произведению старших членов этих многочленов. 
\end{lem}
\begin{proof}[\textbf{Доказательство}]
Пусть $f, g$ - рассматриваемые многочлены, $A, B$ - их старшие члены соответственно, $a \neq A$ - член многочлена $f$, $b \neq B$ - член многочлена $g$. \\
 \indent Старшим членом может быть один из 4 вариантов: $AB, Ab, aB, ab$.
$$B \succ b \Rightarrow AB \succ Ab.$$
$$A \succ a \Rightarrow AB \succ aB.$$
$$B \succ b \Rightarrow aB \succ ab \Rightarrow AB \succ ab.$$ \\
\indent Таким образом, $AB$ - старший из вариантов.
\end{proof}

\subsection{Симметрические многочлены}
\begin{defn}
Многочлен называется \textbf{однородным}, если все его члены имеют одинаковую полную степень.
\end{defn}
\begin{defn}
Многочлен называется \textbf{симметрическим}, если он не меняется при произвольной перестановке его переменных.
\end{defn}
$$f(x_{\varphi(1)}, x_{\varphi(2)}, ..., x_{\varphi(n)}) = f(x_1, x_2, ..., x_n)~ \forall \varphi \in S_n$$
\indent Следующие многочлены называются \textbf{элементарными симметрическими}:
$$\sigma_1 = x_1 + x_2 + ... + x_n$$
$$\sigma_2 = x_1x_2 + ... + x_{n-1}x_n$$
$$\sigma_3 = x_1x_2x_3 + ...$$
$$...$$
$$\sigma_n = x_1x_2...x_n$$
\\
\indent Пусть $f(x) = a_0x^n + a_1x^{n-1} + ... + a_n$, $\alpha_1, \alpha_2, ..., \alpha_n$ - его корни. Тогда:
$$\alpha_1 + \alpha_2 + ... + \alpha_n = -\frac{a_1}{a_0}$$
$$\alpha_1\alpha_2 + ... + \alpha_{n-1}\alpha_n = \frac{a_2}{a_0}$$
$$...$$
$$\alpha_1\alpha_2...\alpha_{n-1}\alpha_n = (-1)^n\frac{a_n}{a_0}$$

\indent Под $S(x_1^{k_1}x_2^{k_2}...x_n^{k_n})$ понимают сумму всех различных членов, полученных всевозможными перестановками. Например, $\sigma_1 = S(x_1), \sigma_2 = S(x_1x_2), \sigma_3 = S(x_1x_2x_3)$.
$$S(x_1^2x_2) = x_1^2x_2 + x_1^2x_3 + x_2^2x_1 + x_2^2x_3 + x_3^2x_1 + x_3^2x_2.$$

\begin{thm}
Любой симметрический многочлен может быть представлен в виде многочлена от элементарных симметрических многочленов, причем это представление единственное.
\end{thm}
\begin{proof}[\textbf{Доказательство}]
Будем доказывать только существование. \\
\indent Предполагаем, что члены лексикографически упорядочены. Пусть старшим членом является $ax_1^{k_1}...x_n^{k_n}$.
Показатели при неизвестных в этом члене должны удовлетворять следующим неравенствам:
$$k_1 \geq k_2 \geq ... \geq k_n$$
\indent Допустим, что $k_1 < k_2$. Но тогда в многочлене существует член $ax_1^{k_2}x_2^{k_1}...x_n^{k_n}$, который старше, чем $ax_1^{k_1}...x_n^{k_n}$. Получено противоречие.
$$\varphi_1(\sigma_1, ..., \sigma_n) = a\sigma_1^{k_1 - k_2}\sigma_2^{k_2 - k_3}...\sigma_{n-1}^{k_{n-1} - k_n}\sigma_n^{k_n}$$
\indent Старшим членом $\sigma_1$ является $x_1$, $\sigma_2 - x_1x_2$, $\sigma_3 - x_1x_2x_3$, ..., $\sigma_n - x_1x_2...x_n$. Тогда старшим членом $\varphi_1$ является: 
$$ax_1^{k_1-k_2}(x_1x_2)^{k_2-k_3}(x_1x_2x_3)^{k_3 - k_4}...(x_1x_2...x_n)^{k_n} = ax_1^{k_1 - k_2 + k_2 - k_3 + ... + k_n}... = ax_1^{k_1}x_2^{k_2}...x_n^{k_n}.$$
\indent Старшие члены $f$ и $\varphi_1$ совпадают. Построим следюущий многочлен:
$$f(x_1, ..., x_n) - \varphi_1(\sigma_1, ..., \sigma_n) = \underbrace{bx_1^{l_1}x_2^{l_2}...x_n^{l_n} + ...}_{\text{старший член, младше $ax_1^{k_1}x_2^{k_2}...x_n^{k_n}$}} $$
\indent Старшие члены многочленов уничтожились, $l_1 \geq l_2 \geq ... \geq l_n$. Проведем аналогичные действия:
$$\varphi_2(\sigma_1, ..., \sigma_n) = b\sigma_1^{l_1 - l_2}\sigma_2^{l_2 - l_3}...\sigma_{n-1}^{l_{n-1} - l_n}\sigma_n^{l_n}$$
\indent Утверждается, что процесс непременно оборвется. Количество последовательностей, удовлетворяющих следующим условиям, конечно:
$$k_1 \geq k_2 \geq ... \geq k_n$$
$$l_1 \geq l_2 \geq ... \geq l_n$$
$$k_1 \geq l_1$$
\indent Даже в случае, когда выполняются первое и третье условия, а также $\forall i~l_i \leq k_1$, количество наборов чисел $l_i$  ограничено числом $(k_1 + 1)^n$.
\end{proof} 
\begin{examp}
Выразить через элементарные симметрические многочлены многочлен $f = x_1^4 + x_2^4 + x_3^4 - 2x_1^2x_2^2 - 2x_1^2x_3^2 - 2x_2^2x_3^2$. \\ \\
\indent Старший член - $x_1^4$. \\ \\
$\begin{array}{c|c|c}
  k_1 & k_2 & k_3 \\ \hline
  4 & 0 & 0 \\
  3 & 1 & 0 \\
  2 & 1 & 1 \\
  2 & 2 & 0
\end{array}$ \\ \\
$$f = \sigma_1^{4-0}\sigma_2^{0-0}\sigma_3^{0} + A\sigma_1^{3-1}\sigma_2^{1-0}\sigma_3^{0} + B\sigma_1^{2-1}\sigma_2^{1-1}\sigma_3^{1} + C\sigma_1^{2-2}\sigma_2^{2-0}\sigma_3^{0} = \sigma_1^{4} + A\sigma_1^{2}\sigma_2 + B\sigma_1\sigma_3 + C\sigma_2^{2}$$
$$x_1 = 1, x_2 = 1, x_3 = 1 \Rightarrow f(x_1,x_2,x_3) = -3, \sigma_1 = x_1 + x_2 + x_3 = 3,$$
$$\sigma_2 = x_1x_2 + x_1x_3 + x_2x_3 = 3, \sigma_3 = x_1x_2x_3 = 1.$$
$$-3 = 81 + 27A + 3B + 9C$$
$$x_1 = 1, x_2 = 1, x_3 = 0 \Rightarrow f(x_1,x_2,x_3) = 0, \sigma_1 = x_1 + x_2 + x_3 = 2,$$
$$\sigma_2 = x_1x_2 + x_1x_3 + x_2x_3 = 1, \sigma_3 = x_1x_2x_3 = 0.$$
$$0 = 16 + 4A + C$$
$$x_1 = 1, x_2 = -1, x_3 = 0 \Rightarrow f(x_1,x_2,x_3) = 0, \sigma_1 = x_1 + x_2 + x_3 = 0,$$
$$\sigma_2 = x_1x_2 + x_1x_3 + x_2x_3 = -1, \sigma_3 = x_1x_2x_3 = 0.$$
$$0 = C \Rightarrow A = -4, B = 8.$$
Ответ: $f = \sigma_1^4 - 4\sigma_1^2\sigma_2 + 8\sigma_1\sigma_3$.
\end{examp}
\begin{examp}
Выразить через элементарные симметрические многочлены многочлен $f = (2x_1 - x_2 - x_3)(2x_2 - x_1 - x_3)(2x_3 - x_1 - x_2)$. \\ \\
\indent Старший член - $2x_1^3$. \\ \\
$\begin{array}{c|c|c}
  k_1 & k_2 & k_3 \\ \hline
  3 & 0 & 0 \\
  2 & 1 & 0 \\
  1 & 1 & 1
\end{array}$ \\ \\
$$f = 2\sigma_1^3 + A\sigma_1\sigma_2 + B\sigma_3$$
$$x_1 = 1, x_2 = 1, x_3 = 1 \Rightarrow f(x_1,x_2,x_3) = 0, \sigma_1 = x_1 + x_2 + x_3 = 3,$$
$$\sigma_2 = x_1x_2 + x_1x_3 + x_2x_3 = 3, \sigma_3 = x_1x_2x_3 = 1.$$
$$0 = 54 + 9A + B \Rightarrow B = -9A - 54$$
$$x_1 = 1, x_2 = 1, x_3 = 0 \Rightarrow f(x_1,x_2,x_3) = -2, \sigma_1 = x_1 + x_2 + x_3 = 2,$$
$$\sigma_2 = x_1x_2 + x_1x_3 + x_2x_3 = 1, \sigma_3 = x_1x_2x_3 = 0.$$
$$-2 = 16 + 2A \Rightarrow A = -9, B = 27$$
Ответ: $f = 2\sigma_1^3  - 9\sigma_1^2\sigma_2 + 27\sigma_3$.
\end{examp}

\subsection{Степенные суммы}
$$S_1 = x_1 + x_2 + x_3 + ... + x_n.$$
$$S_2 = x^2_1 + x^2_2 + x^2_3 + ... + x^2_n.$$
$$S_3 = x^3_1 + x^3_2 + x^3_3 + ... + x^3_n.$$
$$...$$
\indent Удобно ввести $S_0 = n$ (сумма нулевых степеней, то есть единиц).\\ \\
\indent Выведем реккурентные соотношения для нахождения $S_n$. Рассмотрим многочлен: $$f(x) = (x-x_1)(x-x_2)...(x-x_n)$$
\indent По формуле Виета:
$$f(x) = x^n - \sigma_1x^{n-1} + \sigma_2x^{n-2} - ... + (-1)^n\sigma_n.$$
$$f'(x) = nx^{n-1} - (n-1)\sigma_1x^{n-2} + (n-2)\sigma_2x^{n-3} - ... \eqno(1)$$
\indent Найдем производную иначе:
$$(uv)' = u'v + uv'$$
$$(uvw)' = u'vw + uv'w + uvw'$$
$$f'(x) = (x-x_2)(x-x_3)...(x-x_n) + (x-x_1)(x-x_3)...(x-x_n) + ...$$
$$f'(x) = \sum\limits_{i=1}^n \frac{f(x)}{x - x_i}$$
$\begin{array}{c|c|c|c|c|c|c}
  & 1 & -\sigma_1 & \sigma_2 & -\sigma_3 & ... & (-1)^n \\ \hline
x_i & 1 & x_i - \sigma_1 & x_i^2 - \sigma_1x_i + \sigma_2 & x_i^3 - \sigma_1x_i^2 + \sigma_2x_i - \sigma_3 & & 0 \\
\end{array}$
\\ \\
\indent Коэффициент в $i$-ом слагаемом при $x^{n-1-k}$: $x_i^k - \sigma_1x_i^{k-1} + \sigma_2x_i^{k-2} - ... + (-1)^k\sigma_k$,
$k = 1,2,...,n-1.$ Для того, чтобы получить коэффициент при $x^{n-1-k}$ во всей $f'(x)$, необходимо просуммировать:
$$S_k - \sigma_1S_{k-1} + \sigma_2S_{k-2} - ... + (-1)^kn\sigma_k \eqno(2)$$
\indent Приравняем коэффициенты из $(1)$ и $(2)$. Коэффициент при $k = 1$ в $(1)$ равен $-(n-1)\sigma_1$, при $k = 2$ равен $(n-2)\sigma_2$, при  $k = 3$ равен $-(n-3)\sigma_3$. Тогда получим формулу:
$$S_k - \sigma_1S_{k-1} + \sigma_2S_{k-2} - ... + (-1)^kn\sigma_k = (-1)^k(n-k)\sigma_k$$
$$(-1)^k\sigma_k(n-(n-k)) = (-1)^kk\sigma_k \Rightarrow $$
$$\Rightarrow S_k - \sigma_1S_{k-1} + \sigma_2S_{k-2} - ... + (-1)^kk\sigma_k = 0,~k = 1,2,...,n-1.$$
\indent Но эта формула интересна только в случае $k > 1$.\\ \\
\indent Рассмотрим случай $k = n$.
$$f(x) = x^n - \sigma_1x^{n-1} + \sigma_2x^{n-2} - ... + (-1)^n\sigma_n$$
$$f(x_i) = 0~\forall i \Rightarrow x_i^n - \sigma_1x_i^{n-1} + \sigma_2x_i^{n-2} - ... + (-1)^n\sigma_n = 0~\forall i$$
\indent Вновь просуммируем по $i$, тогда получим:
$$S_n - \sigma_1S_{n-1} + \sigma_2S_{n-2} - ... + (-1)^nn\sigma_n = 0$$
\indent Таким образом, формула для случая $k < n$ работает и для $k = n$. \\ \\ 
\indentТеперь рассмотрим случай $k > n$.
$$x_i^n - \sigma_1x_i^{n-1} + \sigma_2x_i^{n-2} - ... + (-1)^n\sigma_n = 0~~|\cdot x_i^{k-n}$$
$$x_i^{k} - \sigma_1x_i^{k-1} + \sigma_2x_i^{k-2} - ... + (-1)^nx_i^{k-n}\sigma_n = 0$$
\indentПосле суммирования по $i$ получим следующую формулу:
$$S_k - \sigma_1S_{k-1} + ... + (-1)^nS_{k-n}\sigma_n = 0$$

\begin{examp}
Найти формулы для $S_2$ и $S_3$.
\begin{itemize}
\item[$1)$]
Так как $k = 2, n \geq 2$, можно использовать первую формулу:
$$S_2 - \sigma_1S_1 + 2\sigma_2 = 0$$
$$S_1 = \sigma_1 \Rightarrow S_2 = \sigma_1^2 - 2\sigma_2$$
\item[$2.1)$]
$k = 3$, если $n \geq 3 \Rightarrow k \leq n$.
$$S_3 - \sigma_1S_2 + \sigma_2S_1 - 3\sigma_3 = 0$$
$$S_3 = \sigma_1S_2 - \sigma_2S_1 + 3\sigma_3 = \sigma_1(\sigma_1^2 - 2\sigma_2) - \sigma_2\sigma_1 + 3\sigma_3 = 
\sigma_1^3 - 3\sigma_1\sigma_2 + 3\sigma_3$$
\item[$2.2)$]
$k = 3$, если $n = 2$.
$$S_3 - \sigma_1S_2 + \sigma_2S_3 = 0 \Rightarrow S_3 = \sigma_1S_2 - \sigma_2S_3 = \sigma_1^3  - 3\sigma_1\sigma_2$$
\end{itemize}
\indent На самом деле, случай 2.2 подпадает под 2.1, так как в 2.2 $\sigma_3 = 0$.
\end{examp}

\end{document}