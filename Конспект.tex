\documentclass[12pt]{article}
\usepackage[utf8]{inputenc}
\usepackage[T2A]{fontenc}
\usepackage[russian, english]{babel}
\usepackage{amsthm}
\usepackage{amssymb}
\usepackage{amsmath}
\usepackage{authblk}
\usepackage{bm}
\usepackage{enumerate}
\usepackage{mathabx}
\usepackage{titling}
\usepackage{titlesec}
\usepackage{ragged2e}
\usepackage{polynom}
\usepackage{hyperref}
\usepackage[normalem]{ulem}
\usepackage[left=3cm,right=3cm, top=2cm,bottom=2cm,bindingoffset=0cm]{geometry}

\titlelabel{\thetitle.\quad}

\hypersetup{
    colorlinks=true, %set true if you want colored links
    linktoc=all,     %set to all if you want both sections and subsections linked
    linkcolor=black,  %choose some color if you want links to stand out
}
\textwidth = 16cm

\newcommand{\stkout}[1]{\ifmmode\text{\sout{\ensuremath{#1}}}\else\sout{#1}\fi}
\hyphenation{ко-эф-фи-ци-ент со-мно-жи-те-лей пер-вой вы-чис-ле-ни-я-ми мно-го-чле-нов
про-из-ве-де-ни-е мно-жи-те-ля раз-ло-же-ни-е мно-го-чле-на}

\begin{document}

\renewcommand{\contentsname}{Содержание}
\title{Алгоритмы компьютерной алгебры}
\author{Конспект лекций}
\date{2019}
\maketitle
\newpage

\tableofcontents
\newpage

\theoremstyle{plain}
\newtheorem{thm}{Теорема}
\newtheorem*{rev*}{Обратное утверждение}
\newtheorem{defn}{Определение}
\newtheorem{examp}{Пример}
\newtheorem{lem}{Лемма}
\newtheorem{sled}{Следствие}
\newtheorem*{sl*}{Следствие}
\newtheorem{predp}{Предположение}


\def\MYdeg{\mathop{\rm deg}\limits}

\section{Лекция 1.}
\hspace{0.6cm}Предмет изучения компьютерной алгебры - точные вычисления. Рассматриваются именно алгоритмы точного, а не приближенного вычисления, как в вычислительной математике. Эти алгоритмы лежат в основе математических пакетов MATLAB, Mathematica. Основной объект исследований - числовые системы с точными вычислениями.

\subsection{Основные факты из теории многочленов}
\begin{defn}
\textbf{Числовым полем} называется множество $\mathrm{F} \subset \mathbb{C}$, если:
\begin{enumerate}
\item $0, 1 \in F$,
\item $\mathrm{|F| \ge 2}$,
\item $\forall a,b \in F: a \pm b,~ab \in F;~b \ne 0, \dfrac{a}{b} \in F.$
\end{enumerate}
\end{defn}

\begin{examp}
Числовые поля - $\mathbb{C}, \mathbb{R}, \mathbb{Q}, \{a + b\sqrt{2},~a,b \in \mathbb{Q}\}$
\end{examp}
Множество многочленов над полем рациональных чисел обозначается как $\mathbb{Q}[x]$, над целыми $-~\mathbb{Z}[x]$,
над произвольным числовым полем $F$ $-~F[x].$

\begin{defn}
Многочлен $f(x) \in \mathrm{F}[x]$, отличный от константы, называют \textbf{ приводимым} над полем $\mathrm{F}$, если он допускает представление вида $f(x) = \varphi(x) \psi(x)$, где $\varphi(x), \psi(x) \in \mathrm{F}[x]$ и $\MYdeg \varphi, \MYdeg \psi < \MYdeg f$, и \textbf{ неприводимым}, если он не допускает такого разложения (то есть один из многочленов $\varphi, \psi$ является константой).
\end{defn}

\begin{enumerate}
\item $\MYdeg f = 1$.
Пусть $f$ допускает разложение: $f(x) = \varphi(x) \psi(x)$. 
$$\MYdeg_{= 0}\varphi, \MYdeg_{= 0}\psi < \MYdeg f \Rightarrow \MYdeg f = 0.$$
Полученное противоречие доказывает неприводимость любого многочлена первой степени.
\item Пусть $\MYdeg f > 1$ и $f(\alpha) = 0, \alpha \in F.$
$$(x - \alpha)~|~f(x) \Rightarrow \exists g(x): f(x) = (x - \alpha)g(x).$$
$$\MYdeg (x - \alpha) = 1 < \MYdeg f.$$
$$\MYdeg g = \MYdeg f - 1 < \MYdeg f.$$
Если многочлен $f$ имеет корень в поле $F$, то $f$ приводим над полем $F$. 
\end{enumerate}

\begin{rev*}
Если многочлен $f \in F[x]$ степени 2 или 3 приводим над полем $F$, то он имеет в этом поле корень. 
\end{rev*}
\begin{proof}[\textbf{Доказательство}]
Допустим, многочлен приводим, следовательно, $f(x) = \varphi(x) \psi(x)$.
$$\MYdeg \varphi, \MYdeg \psi < \MYdeg f \Rightarrow \MYdeg \varphi = 1 \text{ или} \MYdeg \psi = 1.$$
Допустим, $\varphi(x) = ax + b, a \neq 0 \Rightarrow \alpha = -\dfrac{b}{a}, \alpha \in F.$
\end{proof}

\begin{examp}
\mbox{}
\begin{enumerate}
\item $f(x) = x^2 - 1 = (x - 1)(x + 1).$
Многочлен приводим над полями $\mathbb{Q}, \mathbb{R}, \mathbb{C}$.
\item $f(x) = x^2 - 2 \in \mathbb{Q}[x]$. У него нет рациональных корней, следовательно, он неприводим над $\mathbb{Q}$. Но $f(x) = (x - \sqrt{2})(x + \sqrt{2}) \Rightarrow f(x)$ приводим над $\mathbb{R}$.
\item $f(x) = x^2 + 1$ неприводим над $\mathbb{Q}$ и $\mathbb{R}$. Но $f(x) = (x - i)(x + i) \Rightarrow f(x)$ приводим над $\mathbb{C}$.
\end{enumerate}
\end{examp}

Многочлены второй и третьей степени приводимы над полем $F$ тогда и только тогда, когда имеют в этом корень. Для многочленов степени, больше чем 3, данное утверждение не является справедливым.

\begin{examp}
$f(x) = (x^2 + 1)^2 \in \mathbb{R}[x]$ не имеет действительных корней, но приводим.
\end{examp}

\begin{defn}
Многочлен называется \textbf{нормированным}, если его старший коэффициент равен единице.
\end{defn}

\begin{thm}[Фундаментальная теорема о многочленах]
Пусть $f \in \mathrm{F}[x],$\mbox{$~deg~f \geq 1$.} Тогда $f$ допускает разложение $f(x) = a_0 \varphi_1(x) \varphi_2(x) ... \varphi_k(x)$,
 где $a_0 \in \mathrm{F},~\varphi_i \in \mathrm{F}[x]$ и любой многочлен $\varphi_i$ - нормированный и неприводимый. При этом данное разложение является единственным с точностью до порядка следования сомножителей.
\end{thm}

\subsection{Многочлены с рациональными коэффициентами}
\hspace{0.6cm}Дан многочлен с рациональными коэффициентами. Задача: найти разложение этого многочлена в произведение многочленов с рациональными коэффициентами.

Пусть $f \in \mathbb{Q}[x].$ Если мы умножим этот многочлен на подходящее число $N$ (наименьшее общее кратное коэффициентов членов многочлена), то $Nf(x) \in \mathbb{Z}[x].$ Таким образом, приводимость $f$ равносильна приводимости $Nf$, следовательно, разложение многочлена с рациональными коэффициентами можно свести к разложению многочлена с целыми коэффициентами. 

\begin{thm}
Если многочлен $f \in \mathbb{Z}[x]$ допускает разложение в произведение многочленов с рациональными коэффициентами, то он допускает разложение в произведение многочленов тех же степеней с целыми коэффициентами.
\end{thm}

\subsubsection{Алгоритм Кронекера}
\hspace{0.6cm}Дан многочлен $f \in \mathbb{Z}[x],~deg~f>1.$ Можно ли подобрать $u(x), v(x), ~u, v \in \mathbb{Z}[x]$ и $deg~u,~deg~v < deg~f?$

\begin{predp}
Все возникающие натуральные числа можно факторизовать.
\end{predp}

\begin{predp}
Многочлен формальной степени $n$ можно найти с помощью интерполяционного многочлена по $n + 1$ точке \mbox{$x_0, x_1, ..., x_{n}$} и значениям многочлена в этих точках \mbox{$f(x_0), f(x_1), ..., f(x_n)$.}
\end{predp}

$$\begin{cases}
f(x_0) = u(x_0)v(x_0), \\
f(x_1) = u(x_1)v(x_1), \\
... \\
f(x_n) = u(x_n)v(x_n). \\
\end{cases}$$

Рассмотрим точки $x_0, x_1, ..., x_{n} \in \mathbb{Z} \Rightarrow \forall i \in [0, n]: f(x_i) \in \mathbb{Z} \Rightarrow u(x_i), v(x_i) \in \mathbb{Z}.$ Пусть все рассматриваемые точки - не корни многочлена $f$. Тогда $u(x_i)~|~f(x_i)$

\subsubsection{Алгоритм Евклида}

\hspace{0.6cm}Если многочлены $f, g \in F[x],~g \neq 0,$ то имеет место следующее представление: \mbox{$f(x) = g(x)h(x) + r(x)$}$,~h, r \in F[x]$ и $r = 0$ или $r \neq 0,~deg~r < deg~g.$
Если считать, что степень нулевого многочлена $r = 0$ равна $-\infty$, то можно рассматривать только вариант $deg~r < deg~g$.

\begin{defn}
Если многочлены $f, g \in F[x]$, то многочлен $\varphi \in F[x]$ называют \textbf{наибольшим общим делителем (НОД)} $f$ и $g$, если:
\begin{enumerate}
\item $\varphi(x)~|~f(x),~ \varphi(x)~|~g(x)$,
\item $\forall \psi \in F[x]: \psi(x)~|~f(x),~ \psi(x)~|~g(x) \Rightarrow \psi(x)~|~\varphi(x)$.
\end{enumerate}
\end{defn}
Можно доказать, что НОД всегда существует и находится с точностью до множителя. Если старши

\begin{examp}
Получить каноническое разложение многочлена $$f(x) = (x - 1)(x - 2)(x^2 + x + 1)^2(x^2 - x + 1)^2(x^3 - 2)^3.$$
$$f(x) = \varphi_1(x)(\varphi_2(x))^2(\varphi_3(x))^3.$$
$$\varphi_1(x) = (x - 1)(x - 2).$$
$$\varphi_2(x) = (x^2 + x + 1)(x^2 - x + 1).$$
$$\varphi_3(x) = x^3 - 2.$$
\end{examp}

\end{document}